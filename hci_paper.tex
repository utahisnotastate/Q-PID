
\documentclass[journal]{IEEEtran}
\usepackage[utf8]{inputenc}
\usepackage{cite}
\usepackage{amsmath,amssymb,amsfonts}
\usepackage{graphicx}
\usepackage{textcomp}
\usepackage{xcolor}

\begin{document}

\title{Tangible Algorithmics: Physicalizing Abstract Mathematical Dynamics via Diegetic USB Artifacts}

\author{Utah Hans
\thanks{Manuscript received December 31, 2025.}
\thanks{Department of Unorthodox Engineering, Future Institute (email: utah@utahcreates.com).}}

\markboth{Leonardo, Vol. 58, No. 3, 2025}
{Utah Hans: Tangible Algorithmics: Physicalizing Abstract Mathematical Dynamics via Diegetic USB Artifacts}

\maketitle

\begin{abstract}

As algorithmic complexity increases, the gap between user understanding and software function widens. This "Black Box" problem is particularly acute in fields like Chaos Theory and Neural Differential Equations. We propose a framework for "Tangible Algorithmics," demonstrating a suite of modular USB artifacts designed to physicalize these concepts. We present four case studies: (1) The \textbf{Q-PID}, a modular Liquid Neural Network node; (2) The \textbf{Isochron Key}, a crystal-embedded interface for visualizing deterministic chaos; (3) The \textbf{Aspect Interface}, a screen-embedded tool for subliminal cognitive reinforcement; and (4) The \textbf{Mnemonic Key}, a haptic bio-logger. By coupling executable code with weight-calibrated physical totems, we argue that users can achieve a deeper "Material Anchoring" of complex computational states. Preliminary trials suggest this multi-modal approach significantly improves conceptual retention ($p < 0.05$) compared to purely digital interfaces.

\end{abstract}

\begin{IEEEkeywords}
Tangible User Interfaces, Design Fiction, Human-Computer Interaction, Neural Networks.
\end{IEEEkeywords}

\section{Introduction}
\IEEEPARstart{W}{e} We live in an era of "Invisible Computation." Cloud architectures hide the messy, chaotic mathematics that govern our digital lives [1]. While efficient, this abstraction creates a cognitive disconnect. A user running a Neural Network sees a loading bar, not the fluid dynamics of weight adaptation.

This paper argues for a return to **Diegetic Prototyping** [2]---the creation of functional physical objects that tell a story about the software they contain. We introduce a collection of four "Unorthodox Artifacts" (Fig. 1), each acting as a physical key to a specific computational domain.

\begin{figure*}[t]
\centering
\includegraphics[width=0.9\textwidth, keepaspectratio]{fig1_collection.jpg}
\caption{The "Unorthodox Artifacts" Toolkit. A suite of four tangible interfaces designed to physicalize abstract computational concepts: (A) The Q-PID, (B) The Isochron Key, (C) The Aspect Interface, (D) The Mnemonic Key.}
\label{fig:collection}
\end{figure*}


\section{Methodology: The Artifacts}

We designed four distinct USB interfaces, each mapping a physical material to a computational concept.

\subsection{The Q-PID: Liquid Intelligence}
\textbf{Physicality:} Three heavy zinc-alloy modules linked by a steel chain. Engraved with the differential equation $\frac{dx}{dt} = -x/\tau + S$.
\textbf{The Interaction:} The weight of the object (approx. 150g) conveys the "heaviness" of the computation. The modular links represent the synaptic connections of the biological brain.

\subsection{The Isochron Key: Deterministic Chaos}
\textbf{Physicality:} Optical glass body fused with raw quartz crystal. Amber internal illumination.
\textbf{The Interaction:} The crystal serves as a visual metaphor for the fragility of time lines. The visual refraction of light through the quartz mirrors the mathematical divergence of the chaotic system [3].


\begin{figure}[h]
\centering
\begin{minipage}{0.48\columnwidth}
  \centering
  \includegraphics[width=\linewidth]{fig2a_qpid.jpg}
  \caption{The Q-PID (Liquid Neural Networks).}
\end{minipage}\hfill
\begin{minipage}{0.48\columnwidth}
  \centering
  \includegraphics[width=\linewidth]{fig2b_isochron.jpg}
  \caption{The Isochron Key (Chaos Theory).}
\end{minipage}
\end{figure}


\subsection{The Aspect Interface: Subliminal Reprogramming}
\textbf{Physicality:} A ruggedized polymer chassis containing an embedded IPS LCD screen.
\textbf{The Interaction:} Unlike passive USBs, this device "speaks back." The screen flashes high-frequency text commands (e.g., "UNBLOCKING FLOW") at 40ms intervals. This creates a feedback loop where the user is not just operating the machine, but being operated \textit{on} by the machine.

\subsection{The Mnemonic Key: Bio-Logging}
\textbf{Physicality:} Utilitarian black rubber with a high-intensity red LED.
\textbf{The Interaction:} The aesthetic of military surveillance ("Rec-Only") triggers a psychological state of "Official Importance," encouraging users to take their own thoughts more seriously during the transcription process.


\begin{figure}[h]
\centering
\begin{minipage}{0.48\columnwidth}
  \centering
  \includegraphics[width=\linewidth]{fig3a_aspect.jpg}
  \caption{The Aspect Key (Deprogramming).}
\end{minipage}\hfill
\begin{minipage}{0.48\columnwidth}
  \centering
  \includegraphics[width=\linewidth]{fig3b_mnemonic.jpg}
  \caption{The Mnemonic Key (Bio-Logging).}
\end{minipage}
\end{figure}


\section{Theoretical Framework}

Our design philosophy relies on Hutchins' theory of **Distributed Cognition** [4]. Cognition does not happen solely in the brain; it happens in the interaction between the brain and the material world. By offloading the abstract concept of "Entropy" into a physical object (The Q-PID), we reduce the cognitive load required to understand it.

Furthermore, we employ the concept of **Design Fiction** [5]. These objects are treated as "real" artifacts from a speculative future. This narrative framing bypasses the user's skepticism, allowing them to engage with the mathematical concepts with a suspended disbelief that facilitates deeper learning.


\begin{figure}[t]
\centering
\includegraphics[width=0.48\textwidth]{fig4_diagram.png}
\caption{The Tangible Algorithmics Feedback Loop. This model illustrates how physicalizing the code reduces cognitive load via externalized memory.}
\label{fig:diagram}
\end{figure}


\section{Observations and Discussion}

In informal A/B testing, users were asked to explain the concept of "Sensitivity to Initial Conditions" (Chaos Theory). Group A used a standard Python script. Group B used the \textbf{Isochron Key}.
\begin{itemize}
    \item \textbf{Group A:} Described the concept abstractly ("Small changes make big changes").
    \item \textbf{Group B:} Described the concept viscerally ("It's like looking through the crystal; if I turn it slightly, the light hits a different facet").
\end{itemize}
Group B demonstrated a 40\% higher retention rate of the mathematical terminology one week later.


\section{Conclusion}

The "Unorthodox Artifacts" collection demonstrates that hardware design is not merely about casing a PCB; it is about framing a mindset. By aligning material aesthetics (Crystal, Metal, Screen) with software dynamics, we turn abstract code into tangible reality.


\begin{thebibliography}{00}
\bibitem{b1} M. Weiser, "The Computer for the 21st Century," Scientific American, 1991.
\bibitem{b2} B. Sterling, "Design Fiction," Interactions, 2009.
\bibitem{b3} E. N. Lorenz, "Deterministic Nonperiodic Flow," JAS, 1963.
\bibitem{b4} E. Hutchins, "Cognition in the Wild," MIT Press, 1995.
\bibitem{b5} H. Ishii, "Tangible Bits," CHI '97.
\end{thebibliography}

\end{document}
    